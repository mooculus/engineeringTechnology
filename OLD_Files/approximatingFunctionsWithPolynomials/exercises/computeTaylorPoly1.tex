\documentclass{ximera}

%\usepackage{todonotes}
%\usepackage{mathtools} %% Required for wide table Curl and Greens
%\usepackage{cuted} %% Required for wide table Curl and Greens
\newcommand{\todo}{}

\usepackage{esint} % for \oiint
\ifxake%%https://math.meta.stackexchange.com/questions/9973/how-do-you-render-a-closed-surface-double-integral
\renewcommand{\oiint}{{\large\bigcirc}\kern-1.56em\iint}
\fi


\graphicspath{
  {./}
  {ximeraTutorial/}
  {basicPhilosophy/}
  {functionsOfSeveralVariables/}
  {normalVectors/}
  {lagrangeMultipliers/}
  {vectorFields/}
  {greensTheorem/}
  {shapeOfThingsToCome/}
  {dotProducts/}
  {partialDerivativesAndTheGradientVector/}
  {../productAndQuotientRules/exercises/}
  {../normalVectors/exercisesParametricPlots/}
  {../continuityOfFunctionsOfSeveralVariables/exercises/}
  {../partialDerivativesAndTheGradientVector/exercises/}
  {../directionalDerivativeAndChainRule/exercises/}
  {../commonCoordinates/exercisesCylindricalCoordinates/}
  {../commonCoordinates/exercisesSphericalCoordinates/}
  {../greensTheorem/exercisesCurlAndLineIntegrals/}
  {../greensTheorem/exercisesDivergenceAndLineIntegrals/}
  {../shapeOfThingsToCome/exercisesDivergenceTheorem/}
  {../greensTheorem/}
  {../shapeOfThingsToCome/}
  {../separableDifferentialEquations/exercises/}
  {vectorFields/}
}

\newcommand{\mooculus}{\textsf{\textbf{MOOC}\textnormal{\textsf{ULUS}}}}

\usepackage{tkz-euclide}\usepackage{tikz}
\usepackage{tikz-cd}
\usetikzlibrary{arrows}
\tikzset{>=stealth,commutative diagrams/.cd,
  arrow style=tikz,diagrams={>=stealth}} %% cool arrow head
\tikzset{shorten <>/.style={ shorten >=#1, shorten <=#1 } } %% allows shorter vectors

\usetikzlibrary{backgrounds} %% for boxes around graphs
\usetikzlibrary{shapes,positioning}  %% Clouds and stars
\usetikzlibrary{matrix} %% for matrix
\usepgfplotslibrary{polar} %% for polar plots
\usepgfplotslibrary{fillbetween} %% to shade area between curves in TikZ
%\usetkzobj{all}
\usepackage[makeroom]{cancel} %% for strike outs
%\usepackage{mathtools} %% for pretty underbrace % Breaks Ximera
%\usepackage{multicol}
\usepackage{pgffor} %% required for integral for loops



%% http://tex.stackexchange.com/questions/66490/drawing-a-tikz-arc-specifying-the-center
%% Draws beach ball
\tikzset{pics/carc/.style args={#1:#2:#3}{code={\draw[pic actions] (#1:#3) arc(#1:#2:#3);}}}



\usepackage{array}
\setlength{\extrarowheight}{+.1cm}
\newdimen\digitwidth
\settowidth\digitwidth{9}
\def\divrule#1#2{
\noalign{\moveright#1\digitwidth
\vbox{\hrule width#2\digitwidth}}}





\newcommand{\RR}{\mathbb R}
\newcommand{\R}{\mathbb R}
\newcommand{\N}{\mathbb N}
\newcommand{\Z}{\mathbb Z}

\newcommand{\sagemath}{\textsf{SageMath}}


%\renewcommand{\d}{\,d\!}
\renewcommand{\d}{\mathop{}\!d}
\newcommand{\dd}[2][]{\frac{\d #1}{\d #2}}
\newcommand{\pp}[2][]{\frac{\partial #1}{\partial #2}}
\renewcommand{\l}{\ell}
\newcommand{\ddx}{\frac{d}{\d x}}

\newcommand{\zeroOverZero}{\ensuremath{\boldsymbol{\tfrac{0}{0}}}}
\newcommand{\inftyOverInfty}{\ensuremath{\boldsymbol{\tfrac{\infty}{\infty}}}}
\newcommand{\zeroOverInfty}{\ensuremath{\boldsymbol{\tfrac{0}{\infty}}}}
\newcommand{\zeroTimesInfty}{\ensuremath{\small\boldsymbol{0\cdot \infty}}}
\newcommand{\inftyMinusInfty}{\ensuremath{\small\boldsymbol{\infty - \infty}}}
\newcommand{\oneToInfty}{\ensuremath{\boldsymbol{1^\infty}}}
\newcommand{\zeroToZero}{\ensuremath{\boldsymbol{0^0}}}
\newcommand{\inftyToZero}{\ensuremath{\boldsymbol{\infty^0}}}



\newcommand{\numOverZero}{\ensuremath{\boldsymbol{\tfrac{\#}{0}}}}
\newcommand{\dfn}{\textbf}
%\newcommand{\unit}{\,\mathrm}
\newcommand{\unit}{\mathop{}\!\mathrm}
\newcommand{\eval}[1]{\bigg[ #1 \bigg]}
\newcommand{\seq}[1]{\left( #1 \right)}
\renewcommand{\epsilon}{\varepsilon}
\renewcommand{\phi}{\varphi}


\renewcommand{\iff}{\Leftrightarrow}

\DeclareMathOperator{\arccot}{arccot}
\DeclareMathOperator{\arcsec}{arcsec}
\DeclareMathOperator{\arccsc}{arccsc}
\DeclareMathOperator{\si}{Si}
\DeclareMathOperator{\scal}{scal}
\DeclareMathOperator{\sign}{sign}


%% \newcommand{\tightoverset}[2]{% for arrow vec
%%   \mathop{#2}\limits^{\vbox to -.5ex{\kern-0.75ex\hbox{$#1$}\vss}}}
\newcommand{\arrowvec}[1]{{\overset{\rightharpoonup}{#1}}}
%\renewcommand{\vec}[1]{\arrowvec{\mathbf{#1}}}
\renewcommand{\vec}[1]{{\overset{\boldsymbol{\rightharpoonup}}{\mathbf{#1}}}}

\newcommand{\point}[1]{\left(#1\right)} %this allows \vector{ to be changed to \vector{ with a quick find and replace
\newcommand{\pt}[1]{\mathbf{#1}} %this allows \vec{ to be changed to \vec{ with a quick find and replace
\newcommand{\Lim}[2]{\lim_{\point{#1} \to \point{#2}}} %Bart, I changed this to point since I want to use it.  It runs through both of the exercise and exerciseE files in limits section, which is why it was in each document to start with.

\DeclareMathOperator{\proj}{\mathbf{proj}}
\newcommand{\veci}{{\boldsymbol{\hat{\imath}}}}
\newcommand{\vecj}{{\boldsymbol{\hat{\jmath}}}}
\newcommand{\veck}{{\boldsymbol{\hat{k}}}}
\newcommand{\vecl}{\vec{\boldsymbol{\l}}}
\newcommand{\uvec}[1]{\mathbf{\hat{#1}}}
\newcommand{\utan}{\mathbf{\hat{t}}}
\newcommand{\unormal}{\mathbf{\hat{n}}}
\newcommand{\ubinormal}{\mathbf{\hat{b}}}

\newcommand{\dotp}{\bullet}
\newcommand{\cross}{\boldsymbol\times}
\newcommand{\grad}{\boldsymbol\nabla}
\newcommand{\divergence}{\grad\dotp}
\newcommand{\curl}{\grad\cross}
%\DeclareMathOperator{\divergence}{divergence}
%\DeclareMathOperator{\curl}[1]{\grad\cross #1}
\newcommand{\lto}{\mathop{\longrightarrow\,}\limits}

\renewcommand{\bar}{\overline}

\colorlet{textColor}{black}
\colorlet{background}{white}
\colorlet{penColor}{blue!50!black} % Color of a curve in a plot
\colorlet{penColor2}{red!50!black}% Color of a curve in a plot
\colorlet{penColor3}{red!50!blue} % Color of a curve in a plot
\colorlet{penColor4}{green!50!black} % Color of a curve in a plot
\colorlet{penColor5}{orange!80!black} % Color of a curve in a plot
\colorlet{penColor6}{yellow!70!black} % Color of a curve in a plot
\colorlet{fill1}{penColor!20} % Color of fill in a plot
\colorlet{fill2}{penColor2!20} % Color of fill in a plot
\colorlet{fillp}{fill1} % Color of positive area
\colorlet{filln}{penColor2!20} % Color of negative area
\colorlet{fill3}{penColor3!20} % Fill
\colorlet{fill4}{penColor4!20} % Fill
\colorlet{fill5}{penColor5!20} % Fill
\colorlet{gridColor}{gray!50} % Color of grid in a plot

\newcommand{\surfaceColor}{violet}
\newcommand{\surfaceColorTwo}{redyellow}
\newcommand{\sliceColor}{greenyellow}




\pgfmathdeclarefunction{gauss}{2}{% gives gaussian
  \pgfmathparse{1/(#2*sqrt(2*pi))*exp(-((x-#1)^2)/(2*#2^2))}%
}


%%%%%%%%%%%%%
%% Vectors
%%%%%%%%%%%%%

%% Simple horiz vectors
\renewcommand{\vector}[1]{\left\langle #1\right\rangle}


%% %% Complex Horiz Vectors with angle brackets
%% \makeatletter
%% \renewcommand{\vector}[2][ , ]{\left\langle%
%%   \def\nextitem{\def\nextitem{#1}}%
%%   \@for \el:=#2\do{\nextitem\el}\right\rangle%
%% }
%% \makeatother

%% %% Vertical Vectors
%% \def\vector#1{\begin{bmatrix}\vecListA#1,,\end{bmatrix}}
%% \def\vecListA#1,{\if,#1,\else #1\cr \expandafter \vecListA \fi}

%%%%%%%%%%%%%
%% End of vectors
%%%%%%%%%%%%%

%\newcommand{\fullwidth}{}
%\newcommand{\normalwidth}{}



%% makes a snazzy t-chart for evaluating functions
%\newenvironment{tchart}{\rowcolors{2}{}{background!90!textColor}\array}{\endarray}

%%This is to help with formatting on future title pages.
\newenvironment{sectionOutcomes}{}{}



%% Flowchart stuff
%\tikzstyle{startstop} = [rectangle, rounded corners, minimum width=3cm, minimum height=1cm,text centered, draw=black]
%\tikzstyle{question} = [rectangle, minimum width=3cm, minimum height=1cm, text centered, draw=black]
%\tikzstyle{decision} = [trapezium, trapezium left angle=70, trapezium right angle=110, minimum width=3cm, minimum height=1cm, text centered, draw=black]
%\tikzstyle{question} = [rectangle, rounded corners, minimum width=3cm, minimum height=1cm,text centered, draw=black]
%\tikzstyle{process} = [rectangle, minimum width=3cm, minimum height=1cm, text centered, draw=black]
%\tikzstyle{decision} = [trapezium, trapezium left angle=70, trapezium right angle=110, minimum width=3cm, minimum height=1cm, text centered, draw=black]


\author{Jim Talamo}
\license{Creative Commons 3.0 By-bC}


\outcome{Compute a Taylor Polynomial}


\begin{document}
\begin{exercise}
Find the third degree Taylor polynomial centered at $x=1$ for the function $f(x) = \sqrt{2x-1}$.


Since we want a Taylor Polynomial centered at $x=1$, we want to look for a polynomial in powers of $x-1$.  Since we want a third degree Taylor polynomial, this means we are looking for a polynomial of the form:

\begin{multipleChoice}
\choice{$p_3(x) = a_0+a_1x+a_2x^2+a_3x^3$}
\choice[correct]{$p_3(x) =  a_0+a_1(x-1)+a_2(x-1)^2+a_3(x-1)^3$}
\end{multipleChoice} 

The coefficients are found using the requirements:

\begin{selectAll}
\choice[correct]{$f(1) = p_3(1)$}
\choice[correct]{$f'(1) = p_3'(1)$}
\choice[correct]{$f''(1) = p_3''(1)$}
\choice[correct]{$f'''(1) = p_3'''(1)$}
\choice{$f^{(4)}(1) = p_3^{(4)}(1)$}
\choice{$f^{(5)}(1) = p_3^{(5)}(1)$}
\end{selectAll}

These requirements are used to establish a formula that relates the coefficients of the Taylor Polynomial to the derivatives of the function that it approximates for each $k=0,1,2,\ldots,n$:

\[
a_k = \frac{f^{(k)}(c)}{k!}
\]
where $c$ is the $x$-value at which the series is centered.  Here, $c=\answer{1}$.  

\begin{exercise}
Complete the table below:

\begin{tabular}{|c|c|c|c|}
\hline
$k$ \quad & \quad \quad $f^{(k)}(x)$  \quad \quad & \quad \quad $f^{(k)}(1)$ \quad \quad & \quad \quad $a_k = \frac{f^{(k)}(1)}{k!}$ \quad \quad \\
\hline 
$0$ \quad & \quad \quad $\answer{(2x-1)^{1/2}}$  \quad \quad & \quad \quad $\answer{1}$ \quad \quad  & \quad \quad $\answer{1}$ \quad \quad \\
\hline
$1$ \quad & \quad \quad $\answer{(2x-1)^{-1/2}}$ \quad \quad & \quad \quad $\answer{1}$ \quad \quad & \quad \quad  $\answer{1}$ \quad \quad  \\
\hline
$2$ \quad & \quad \quad $\answer{-(2x-1)^{-3/2}}$ \quad \quad & \quad \quad $\answer{-1}$ \quad \quad & \quad \quad  $\answer{-\frac{1}{2}}$ \quad \quad  \\
\hline
$3$ \quad & \quad \quad $\answer{3(2x-1)^{-5/2}}$ \quad \quad & \quad \quad $\answer{3}$ \quad \quad & \quad \quad  $\answer{\frac{1}{2}}$ \quad \quad  \\
\hline
\end{tabular}

Hence, the third degree Taylor polynomial for $f(x) =\sqrt{2x-1}$ is:

\begin{align*}
p_3(x) &=a_0+a_1(x-1)+a_2(x-1)^2+a_3(x-1)^3 \\
&= \answer{1}+\answer{1}(x-1)+\answer{-\frac{1}{2}}(x-1)^2+\answer{\frac{1}{2}}(x-1)^3\\
\end{align*}

\end{exercise}

\begin{exercise}
The Taylor polynomials are used to approximate the value of a function near the center.  As long as we are ``close enough" (which we will formalize in a later section), the higher order polynomials should give better approximations.  Suppose that we want to approximate $\sqrt{1.2}$.

The \emph{exact} value is found using the function, provided we use the correct $x$-value.  Since $f(x) = \sqrt{2x-1}$, the $x$-value that gives $\sqrt{1.2}$ is $x=\answer{1.1}$. 

\begin{hint}
To find this, just set $2x-1 = 1.2$.
\end{hint}

Thus, the \emph{exact} answer to 6 decimal places is $\sqrt{1.2} = \answer[tolerance= .000001]{1.095445}$

We can use the third degree Taylor polynomial to write down the first and second degree Taylor polynomials.  We can then substitute $x=1.1$ into these to approximate $\sqrt{1.2}$.  In fact:

\begin{exercise}
The first degree Taylor polynomial is $p_1(x) = \answer{1}+\answer{1}(x-1)$ so to 6 decimal places: $\sqrt{1.2} \approx p_1(1.1) = \answer[tolerance=.000001]{1.100000}$.

The second degree Taylor polynomial is $p_2(x) = \answer{1}+\answer{1}(x-1)+\answer{-\frac{1}{2}}(x-1)^2$ so to 6 decimal places: $\sqrt{1.2} \approx p_2(1.1) = \answer[tolerance=.000001]{1.095000}$.

The third degree Taylor polynomial is $p_3(x) =1+(x-1)-\frac{1}{2}(x-1)^2+\frac{1}{2}(x-1)^3$ so to 6 decimal places: $\sqrt{1.2} \approx p_3(1.1) = \answer[tolerance=.000001]{1.095500}$.

Do the higher order polynomials provide more accurate results?

\begin{multipleChoice}
\choice[correct]{Yes}
\choice{No}
\end{multipleChoice}
\end{exercise}
\end{exercise}

\end{exercise}
\end{document}
