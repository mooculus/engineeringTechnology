\documentclass{ximera}

%\usepackage{todonotes}
%\usepackage{mathtools} %% Required for wide table Curl and Greens
%\usepackage{cuted} %% Required for wide table Curl and Greens
\newcommand{\todo}{}

\usepackage{esint} % for \oiint
\ifxake%%https://math.meta.stackexchange.com/questions/9973/how-do-you-render-a-closed-surface-double-integral
\renewcommand{\oiint}{{\large\bigcirc}\kern-1.56em\iint}
\fi


\graphicspath{
  {./}
  {ximeraTutorial/}
  {basicPhilosophy/}
  {functionsOfSeveralVariables/}
  {normalVectors/}
  {lagrangeMultipliers/}
  {vectorFields/}
  {greensTheorem/}
  {shapeOfThingsToCome/}
  {dotProducts/}
  {partialDerivativesAndTheGradientVector/}
  {../productAndQuotientRules/exercises/}
  {../normalVectors/exercisesParametricPlots/}
  {../continuityOfFunctionsOfSeveralVariables/exercises/}
  {../partialDerivativesAndTheGradientVector/exercises/}
  {../directionalDerivativeAndChainRule/exercises/}
  {../commonCoordinates/exercisesCylindricalCoordinates/}
  {../commonCoordinates/exercisesSphericalCoordinates/}
  {../greensTheorem/exercisesCurlAndLineIntegrals/}
  {../greensTheorem/exercisesDivergenceAndLineIntegrals/}
  {../shapeOfThingsToCome/exercisesDivergenceTheorem/}
  {../greensTheorem/}
  {../shapeOfThingsToCome/}
  {../separableDifferentialEquations/exercises/}
  {vectorFields/}
}

\newcommand{\mooculus}{\textsf{\textbf{MOOC}\textnormal{\textsf{ULUS}}}}

\usepackage{tkz-euclide}\usepackage{tikz}
\usepackage{tikz-cd}
\usetikzlibrary{arrows}
\tikzset{>=stealth,commutative diagrams/.cd,
  arrow style=tikz,diagrams={>=stealth}} %% cool arrow head
\tikzset{shorten <>/.style={ shorten >=#1, shorten <=#1 } } %% allows shorter vectors

\usetikzlibrary{backgrounds} %% for boxes around graphs
\usetikzlibrary{shapes,positioning}  %% Clouds and stars
\usetikzlibrary{matrix} %% for matrix
\usepgfplotslibrary{polar} %% for polar plots
\usepgfplotslibrary{fillbetween} %% to shade area between curves in TikZ
%\usetkzobj{all}
\usepackage[makeroom]{cancel} %% for strike outs
%\usepackage{mathtools} %% for pretty underbrace % Breaks Ximera
%\usepackage{multicol}
\usepackage{pgffor} %% required for integral for loops



%% http://tex.stackexchange.com/questions/66490/drawing-a-tikz-arc-specifying-the-center
%% Draws beach ball
\tikzset{pics/carc/.style args={#1:#2:#3}{code={\draw[pic actions] (#1:#3) arc(#1:#2:#3);}}}



\usepackage{array}
\setlength{\extrarowheight}{+.1cm}
\newdimen\digitwidth
\settowidth\digitwidth{9}
\def\divrule#1#2{
\noalign{\moveright#1\digitwidth
\vbox{\hrule width#2\digitwidth}}}





\newcommand{\RR}{\mathbb R}
\newcommand{\R}{\mathbb R}
\newcommand{\N}{\mathbb N}
\newcommand{\Z}{\mathbb Z}

\newcommand{\sagemath}{\textsf{SageMath}}


%\renewcommand{\d}{\,d\!}
\renewcommand{\d}{\mathop{}\!d}
\newcommand{\dd}[2][]{\frac{\d #1}{\d #2}}
\newcommand{\pp}[2][]{\frac{\partial #1}{\partial #2}}
\renewcommand{\l}{\ell}
\newcommand{\ddx}{\frac{d}{\d x}}

\newcommand{\zeroOverZero}{\ensuremath{\boldsymbol{\tfrac{0}{0}}}}
\newcommand{\inftyOverInfty}{\ensuremath{\boldsymbol{\tfrac{\infty}{\infty}}}}
\newcommand{\zeroOverInfty}{\ensuremath{\boldsymbol{\tfrac{0}{\infty}}}}
\newcommand{\zeroTimesInfty}{\ensuremath{\small\boldsymbol{0\cdot \infty}}}
\newcommand{\inftyMinusInfty}{\ensuremath{\small\boldsymbol{\infty - \infty}}}
\newcommand{\oneToInfty}{\ensuremath{\boldsymbol{1^\infty}}}
\newcommand{\zeroToZero}{\ensuremath{\boldsymbol{0^0}}}
\newcommand{\inftyToZero}{\ensuremath{\boldsymbol{\infty^0}}}



\newcommand{\numOverZero}{\ensuremath{\boldsymbol{\tfrac{\#}{0}}}}
\newcommand{\dfn}{\textbf}
%\newcommand{\unit}{\,\mathrm}
\newcommand{\unit}{\mathop{}\!\mathrm}
\newcommand{\eval}[1]{\bigg[ #1 \bigg]}
\newcommand{\seq}[1]{\left( #1 \right)}
\renewcommand{\epsilon}{\varepsilon}
\renewcommand{\phi}{\varphi}


\renewcommand{\iff}{\Leftrightarrow}

\DeclareMathOperator{\arccot}{arccot}
\DeclareMathOperator{\arcsec}{arcsec}
\DeclareMathOperator{\arccsc}{arccsc}
\DeclareMathOperator{\si}{Si}
\DeclareMathOperator{\scal}{scal}
\DeclareMathOperator{\sign}{sign}


%% \newcommand{\tightoverset}[2]{% for arrow vec
%%   \mathop{#2}\limits^{\vbox to -.5ex{\kern-0.75ex\hbox{$#1$}\vss}}}
\newcommand{\arrowvec}[1]{{\overset{\rightharpoonup}{#1}}}
%\renewcommand{\vec}[1]{\arrowvec{\mathbf{#1}}}
\renewcommand{\vec}[1]{{\overset{\boldsymbol{\rightharpoonup}}{\mathbf{#1}}}}

\newcommand{\point}[1]{\left(#1\right)} %this allows \vector{ to be changed to \vector{ with a quick find and replace
\newcommand{\pt}[1]{\mathbf{#1}} %this allows \vec{ to be changed to \vec{ with a quick find and replace
\newcommand{\Lim}[2]{\lim_{\point{#1} \to \point{#2}}} %Bart, I changed this to point since I want to use it.  It runs through both of the exercise and exerciseE files in limits section, which is why it was in each document to start with.

\DeclareMathOperator{\proj}{\mathbf{proj}}
\newcommand{\veci}{{\boldsymbol{\hat{\imath}}}}
\newcommand{\vecj}{{\boldsymbol{\hat{\jmath}}}}
\newcommand{\veck}{{\boldsymbol{\hat{k}}}}
\newcommand{\vecl}{\vec{\boldsymbol{\l}}}
\newcommand{\uvec}[1]{\mathbf{\hat{#1}}}
\newcommand{\utan}{\mathbf{\hat{t}}}
\newcommand{\unormal}{\mathbf{\hat{n}}}
\newcommand{\ubinormal}{\mathbf{\hat{b}}}

\newcommand{\dotp}{\bullet}
\newcommand{\cross}{\boldsymbol\times}
\newcommand{\grad}{\boldsymbol\nabla}
\newcommand{\divergence}{\grad\dotp}
\newcommand{\curl}{\grad\cross}
%\DeclareMathOperator{\divergence}{divergence}
%\DeclareMathOperator{\curl}[1]{\grad\cross #1}
\newcommand{\lto}{\mathop{\longrightarrow\,}\limits}

\renewcommand{\bar}{\overline}

\colorlet{textColor}{black}
\colorlet{background}{white}
\colorlet{penColor}{blue!50!black} % Color of a curve in a plot
\colorlet{penColor2}{red!50!black}% Color of a curve in a plot
\colorlet{penColor3}{red!50!blue} % Color of a curve in a plot
\colorlet{penColor4}{green!50!black} % Color of a curve in a plot
\colorlet{penColor5}{orange!80!black} % Color of a curve in a plot
\colorlet{penColor6}{yellow!70!black} % Color of a curve in a plot
\colorlet{fill1}{penColor!20} % Color of fill in a plot
\colorlet{fill2}{penColor2!20} % Color of fill in a plot
\colorlet{fillp}{fill1} % Color of positive area
\colorlet{filln}{penColor2!20} % Color of negative area
\colorlet{fill3}{penColor3!20} % Fill
\colorlet{fill4}{penColor4!20} % Fill
\colorlet{fill5}{penColor5!20} % Fill
\colorlet{gridColor}{gray!50} % Color of grid in a plot

\newcommand{\surfaceColor}{violet}
\newcommand{\surfaceColorTwo}{redyellow}
\newcommand{\sliceColor}{greenyellow}




\pgfmathdeclarefunction{gauss}{2}{% gives gaussian
  \pgfmathparse{1/(#2*sqrt(2*pi))*exp(-((x-#1)^2)/(2*#2^2))}%
}


%%%%%%%%%%%%%
%% Vectors
%%%%%%%%%%%%%

%% Simple horiz vectors
\renewcommand{\vector}[1]{\left\langle #1\right\rangle}


%% %% Complex Horiz Vectors with angle brackets
%% \makeatletter
%% \renewcommand{\vector}[2][ , ]{\left\langle%
%%   \def\nextitem{\def\nextitem{#1}}%
%%   \@for \el:=#2\do{\nextitem\el}\right\rangle%
%% }
%% \makeatother

%% %% Vertical Vectors
%% \def\vector#1{\begin{bmatrix}\vecListA#1,,\end{bmatrix}}
%% \def\vecListA#1,{\if,#1,\else #1\cr \expandafter \vecListA \fi}

%%%%%%%%%%%%%
%% End of vectors
%%%%%%%%%%%%%

%\newcommand{\fullwidth}{}
%\newcommand{\normalwidth}{}



%% makes a snazzy t-chart for evaluating functions
%\newenvironment{tchart}{\rowcolors{2}{}{background!90!textColor}\array}{\endarray}

%%This is to help with formatting on future title pages.
\newenvironment{sectionOutcomes}{}{}



%% Flowchart stuff
%\tikzstyle{startstop} = [rectangle, rounded corners, minimum width=3cm, minimum height=1cm,text centered, draw=black]
%\tikzstyle{question} = [rectangle, minimum width=3cm, minimum height=1cm, text centered, draw=black]
%\tikzstyle{decision} = [trapezium, trapezium left angle=70, trapezium right angle=110, minimum width=3cm, minimum height=1cm, text centered, draw=black]
%\tikzstyle{question} = [rectangle, rounded corners, minimum width=3cm, minimum height=1cm,text centered, draw=black]
%\tikzstyle{process} = [rectangle, minimum width=3cm, minimum height=1cm, text centered, draw=black]
%\tikzstyle{decision} = [trapezium, trapezium left angle=70, trapezium right angle=110, minimum width=3cm, minimum height=1cm, text centered, draw=black]

\author{Jim Talamo}

\title[Dig-In:]{Surface areas of revolution}

\outcome{Know the definition of a frustum.}
\outcome{Know how to compute the surface area of a frustum.}
\outcome{Set-up an integral to compute the area of a surface of revolution in terms of $x$.}
\outcome{Set-up an integral to compute the area of a surface of revolution in terms of $y$.}
\outcome{Compute the area of a surface of revolution.}


\begin{document}
\begin{abstract}
We compute surface area of a frustrum then use the method of ``Slice, Approximate, Integrate" to find areas of surface areas of revolution.
\end{abstract}
\maketitle

%% Adoped from APEX

We have already seen how a curve described by $y=f(x)$ on $[a,b]$ can be revolved around an axis to form a solid. Another geometric quantity of interest is the surface area of this solid.  We will be able to apply the procedure of ``Slice, Approximate, Integrate" to find the surface area. 
\section{The area of a frustum}
 In order to perform the approximation step, we first need to discuss the surface area of a \emph{frustrum}.

\begin{definition}
  A \dfn{frustum} of a cone is a section of a cone bounded by two
  planes, where both planes are perpendicular to the height of the
  cone.


%% Herb Clemens! IT IS BEAUTIFUL
To compute the area of a \dfn{surface of revolution}, we approximate that this area is equal to the sum of areas of basic shapes  that we can lay out flat. The argument for this goes way back to the great physicist and mathematician, \index{Archimedes of Alexandria}\textit{Archimedes of Alexandria}. To follow his argument, we have to begin by computing the area of a `lamp shade' or \textit{frustum}.

  \begin{image}[1in]
    \begin{tikzpicture}
      \draw[penColor,very thick,] (-1,0) arc (180:360:1cm and 0.5cm);
      \draw[penColor,very thick,] (-1,0) arc (180:0:1cm and 0.5cm);
      \draw[penColor,very thick,] (-2,-3) arc (180:370:2cm and 1cm);
      \draw[penColor,very thick,dashed] (-2,-3) arc (180:10:2cm and 1cm);
      \draw[penColor,very thick,](-2,-2.9)  -- (-1,0);
      \draw[penColor,very thick,](2,-2.9)   -- (1,0);
      %\node [left] at (2,-1) {$s$};
      \shade[left color=blue!5!white,right color=blue!60!white,opacity=0.3] (-1,0) arc (180:360:1cm and 0.5cm) -- (2,-3) arc (360:180:2cm and 1cm) -- cycle;
      \shade[left color=blue!60!black,right color=blue!5!black,opacity=0.5] (0,0) circle (1cm and 0.5cm);
    \end{tikzpicture}
  \end{image}
\end{definition}

And of course, few things are more interesting than the area of a frustum:

\begin{theorem}
  The surface area of the frustum; meaning the lateral sides, not the top nor the bottom;
  \begin{image}[1in]
    \begin{tikzpicture}
      \draw[penColor,very thick,] (-1,0) arc (180:360:1cm and 0.5cm);
      \draw[penColor,very thick,] (-1,0) arc (180:0:1cm and 0.5cm);
      \draw[penColor,very thick,] (-2,-3) arc (180:370:2cm and 1cm);
      \draw[penColor,very thick,dashed] (-2,-3) arc (180:10:2cm and 1cm);
      \draw[penColor,very thick,] (-2,-2.9)  -- (-1,0);
      \draw[penColor,very thick,] (2,-2.9)   -- (1,0);
      
      \draw[penColor,very thick,dashed] (0,0)   -- (1,0);
      \draw[penColor,very thick,dashed] (0,-3)   -- (2,-3);

      %\draw[decoration={brace,raise=.1cm},decorate,thin] (0,0)   -- (1,0);
      %\draw[decoration={brace,raise=.1cm},decorate,thin] (0,-3)   -- (2,-3);
      \draw[decoration={brace,mirror,raise=.1cm},decorate,thin] (2,-2.9)--(1,0);
      
      \node [right] at (1.7,-1.4) {$s$};
      \node [above] at (.5,0) {$r$};
      \node [above] at (1,-3) {$R$};               
    \end{tikzpicture}
  \end{image}
  is given by
  \[
  \text{area of frustum} = 2\pi\cdot \frac{r+R}{2}\cdot s.
  \]
  \begin{explanation} %Move to Exercises
    We can think of a frustum as approximated by an arrangement of $n$
    congruent trapezoids:
    \begin{image}[1in]
      \begin{tikzpicture}
        
      \draw [penColor,very thick,domain=0:360, samples=11] 
      plot ({cos(\x)}, {.5*sin(\x)} );

      \draw [penColor,very thick,domain=0:360, samples=11] 
        plot ({2*cos(\x)}, {sin(\x)-3} );

      \foreach \x in {0,36,72,...,360}
               {
                 \draw[penColor,very thick,] ({cos(\x)},{.5*sin(\x)}) -- ({2*cos(\x)},{sin(\x)-3}) ;
               }
      \end{tikzpicture}
    \end{image}
    To proceed, we make the following conventions:
    \begin{itemize}
    \item $n$ denote the number of trapezoids,
    \item $t_{n}$ denote the length of the top of each trapezoid,
    \item $h_{n}$ denote the height of each trapezoid,
    \item $b_{n}$ denote the length of the bottom of each trapezoid,
    \end{itemize}
    then from geometry, we have that each of the trapezoids, one of which is shown below:
    \begin{image}[1in]
      \begin{tikzpicture}
        
        \draw [gray,domain=0:360, samples=11] 
        plot ({cos(\x)}, {.5*sin(\x)} );
        
        \draw [gray,domain=0:360, samples=11] 
        plot ({2*cos(\x)}, {sin(\x)-3} );
        
        \foreach \x in {0,36,72,...,360}
                 {
                   \draw[gray] ({cos(\x)},{.5*sin(\x)}) -- ({2*cos(\x)},{sin(\x)-3}) ;
                 }
                 \draw[penColor,very thick]
                 ({cos(252)}, {.5*sin(252)}) --  ({cos(288)}, {.5*sin(288)}) --
                 ({2*cos(288)}, {sin(288)-3}) -- ({2*cos(252)}, {sin(252)-3}) -- ({cos(252)}, {.5*sin(252)});
                 
                 \draw[penColor,very thick,dashed]
                 ({cos(288)}, {.5*sin(288)}) --({cos(288)}, {sin(288)-3});

                 \node [below] at ({(cos(288)+cos(252))},-4) {$b_n$};
                 \node [above] at ({(.25*cos(288)+.25*cos(252))},-.5) {$t_n$};
                 \node [left] at ({cos(288)},-2.25) {$h_n$};
      \end{tikzpicture}
    \end{image}
    has area:
    \[
    \left(  \frac{t_{n}+b_n}{2}\right)\cdot h_n.
    \]
    The area of the frustum approximated by the sum of these $n$
    trapezoids, so the area of the frustum is approximated by
    \[
    A_n = \answer[given]{n}\cdot\left(\frac{t_{n}+b_n}{2}\right)\cdot h_n
    =\left(\frac{n\cdot t_{n}+n\cdot b_n}{2}\right)\cdot h_n.
    \]
    As $n$ goes to infinity, the congruent trapezoids ``smooth'' out:
    \begin{image}
      \begin{tikzpicture}
        
      \draw [penColor,very thick,domain=0:360, samples=11] 
      plot ({cos(\x)}, {.5*sin(\x)} );

      \draw [penColor,very thick,domain=0:360, samples=11] 
        plot ({2*cos(\x)}, {sin(\x)-3} );

      \foreach \x in {0,36,72,...,360}
               {
                 \draw[penColor,very thick,] ({cos(\x)},{.5*sin(\x)}) -- ({2*cos(\x)},{sin(\x)-3}) ;
               }
               

      \draw[penColor,very thick,] (4,0) arc (180:360:1cm and 0.5cm);
      \draw[penColor,very thick,] (4,0) arc (180:0:1cm and 0.5cm);
      \draw[penColor,very thick,] (3,-3) arc (180:370:2cm and 1cm);
      \draw[penColor,very thick,dashed] (3,-3) arc (180:10:2cm and 1cm);
      \draw[penColor,very thick,] (3,-2.9)  -- (4,0);
      \draw[penColor,very thick,] (7,-2.9)   -- (6,0);

      \draw[decoration={brace,mirror,raise=.1cm},decorate,thin] (7,-2.9)--(6,0);
      \node [right] at (6.7,-1.4) {$s$};

      %\node[scale=2] at (-3,-2) {$\lim_{n\to \infty}$}; 
      \node[scale=2] at (2.5,-2) {$\longrightarrow$};
      \end{tikzpicture}
    \end{image}
    Hence,
    \[
    \lim_{n\to \infty} \left(\left(\frac{n\cdot t_{n}+n\cdot b_n}{2}\right)\cdot h_n\right) = \text{area of the frustum.}
    \]
    However, if
    \begin{itemize}
    \item $c$ is the circumference of the top circle,
    \item $s$ is the slant height of the frustum as shown in the above
      figure, and 
    \item $C$ is the circumference of the bottom circle,    
    \end{itemize}
    then
    \begin{align*}
      \lim_{n\to \infty} n \cdot t_{n}  &  = \answer[given]{c},\\
      \lim_{n\to \infty} h_{n}  &  = \answer[given]{s},\\
      \lim_{n\to \infty} n\cdot b_{n}  &  = \answer[given]{C},
    \end{align*}
    and by way of limit laws\index{limit laws} we find
    \[
    \lim_{n\to \infty} \left(\left(\frac{n\cdot t_{n}+n\cdot b_n}{2}\right)\cdot h_n\right) = \answer[given]{\frac{c + C}{2}}\cdot s.
    \]
    Now, letting
    \begin{itemize}
    \item $r$ be the radius of the circle defining the top of the frustum,
    \item $s$ be the slant height of the frustum, and
    \item $R$ be the radius of the circle defining the base of the frustum,
    \end{itemize}
    we see that:
    \begin{align*}
      \text{area of frustum}  &= \answer[given]{\frac{c + C}{2}\cdot s} \\
      &= \frac{\answer[given]{2\pi r + 2\pi R}}{2}\cdot s\\
      &=2\pi \cdot \frac{r+R}{2}\cdot s.
    \end{align*}
  \end{explanation}
\end{theorem}
Now we are ready to compute the area of a surface of revolution.

\section{The area of a surface of revolution}
\index{area of a surface of revolution}

Let's consider a function $f$ with a continuous derivative, and form a surface of revolution formed by this curve by rotating the portion of the curve from $x=a$ to $x=b$ about the $x$-axis:

%%%%%%%%%%%%%%%%%%%%%%%%
\begin{image}
\begin{tikzpicture}
  \begin{axis}[
      xmin=1.5, xmax=3.5,
      domain=1.5:3.5,
      ymin=-4.5, ymax=4.5,
      clip=false,
      axis lines =center,
      xlabel=$x$, ylabel=$y$, every axis y label/.style={at=(current axis.above origin),anchor=south},
      every axis x label/.style={at=(current axis.right of origin),anchor=west},
      axis on top,
      xtick={2,3},
      xticklabels={$a$,$b$}
    ]
  

    \addplot [penColor,very thick,smooth]{16-19*x+8*x^2-x^3};
    \addplot [penColor2,very thick,smooth,domain=2:3]{16-19*x+8*x^2-x^3};
    \addplot [penColor2,very thick,smooth,domain=2:3]{-16+19*x-8*x^2+x^3};
    
    % top ellipses
     \addplot [penColor2,very thick,smooth,domain=2.9:3.1,samples=100]{sqrt(16-16/.01*(x-3)^2)};
       \addplot [penColor2,very thick,smooth,domain=2.9:3.1,samples=100]{-sqrt(16-16/.01*(x-3)^2)};
       %connect edges of top <-annoying numerical approximation
          \addplot[penColor2, very thick] plot coordinates {(3.1,-.2) (3.1,.2)};  
            
       %bottom ellipses 
       \addplot [penColor2,very thick,smooth,domain=1.9:2,samples=100]{sqrt(4-4/.01*(x-2)^2)};
       \addplot [penColor2,very thick,smooth,domain=1.9:2,samples=100]{-sqrt(4-4/.01*(x-2)^2)};
       \addplot [penColor2,very thick,smooth,domain=2:2.1,samples=100,dashed]{sqrt(4-4/.01*(x-2)^2)};
       \addplot [penColor2,very thick,smooth,domain=2:2.1,samples=100,dashed]{-sqrt(4-4/.01*(x-2)^2)};
       
       
      \addplot [penColor2,very thick,smooth,domain=2:3]{-16+19*x-8*x^2+x^3};

    %fill in                        
            	\addplot [name path=A,domain=2:3,draw=none] {16-19*x+8*x^2-x^3};   
            	\addplot [name path=B,domain=2:3,draw=none] {-16+19*x-8*x^2+x^3};
		\addplot [name path=C,domain=1.9:2.1,draw=none,samples=100] {sqrt(4-4/.01*(x-2)^2)};   
            	\addplot [name path=D,domain=1.9:2.1,draw=none,samples=100] {-sqrt(4-4/.01*(x-2)^2)};
		\addplot [name path=E,domain=2.9:3.1,draw=none,samples=100] {sqrt(16-16/.01*(x-3)^2)};   
            	\addplot [name path=F,domain=2.9:3.1,draw=none,samples=100] {-sqrt(16-16/.01*(x-3)^2)};
            	\addplot [fillp] fill between[of=A and B];
	        \addplot [fillp] fill between[of=C and D];
	        \addplot [fillp!60] fill between[of=E and F];


    
    \node[penColor] at (axis cs:2.5,4) {$y=f(x)$};
  \end{axis}
\end{tikzpicture}
\end{image}

We can find a formula that gives the surface area of this surface of revolution using the procedure of ``Slice, Approximate, Integrate"!

\paragraph{Step 1: Slice}
Since we have the curve to be revolved expressed as a function of $x$, we choose to slice with respect to $x$:

%%%%%%%%%%
\begin{image}
\begin{tikzpicture}
  \begin{axis}[
      xmin=1.5, xmax=3.5,
      domain=1.5:3.5,
      ymin=-4.5, ymax=4.5,
      clip=false,
      axis lines =center,
      xlabel=$x$, ylabel=$y$, every axis y label/.style={at=(current axis.above origin),anchor=south},
      every axis x label/.style={at=(current axis.right of origin),anchor=west},
      axis on top,
      xtick={2,3},
      xticklabels={$a$,$b$}
    ]
  

    \addplot [penColor,very thick,smooth]{16-19*x+8*x^2-x^3};
    \addplot [penColor2,very thick,smooth,domain=2:3]{16-19*x+8*x^2-x^3};
    \addplot [penColor2,very thick,smooth,domain=2:3]{-16+19*x-8*x^2+x^3};
    
    % top ellipses
     \addplot [penColor2,very thick,smooth,domain=2.9:3.1,samples=100]{sqrt(16-16/.01*(x-3)^2)};
       \addplot [penColor2,very thick,smooth,domain=2.9:3.1,samples=100]{-sqrt(16-16/.01*(x-3)^2)};
       %connect edges of top <-annoying numerical approximation
          \addplot[penColor2, very thick] plot coordinates {(3.1,-.2) (3.1,.2)};  
            
       %bottom ellipses 
       \addplot [penColor2,very thick,smooth,domain=1.9:2,samples=100]{sqrt(4-4/.01*(x-2)^2)};
       \addplot [penColor2,very thick,smooth,domain=1.9:2,samples=100]{-sqrt(4-4/.01*(x-2)^2)};
       \addplot [penColor2,very thick,smooth,domain=2:2.1,samples=100,dashed]{sqrt(4-4/.01*(x-2)^2)};
       \addplot [penColor2,very thick,smooth,domain=2:2.1,samples=100,dashed]{-sqrt(4-4/.01*(x-2)^2)};
       
       %intermediate ellipses
       \addplot [gray!140,very thick,smooth,domain=2.2:2.3,samples=100]{sqrt(6.017-6.017/.01*(x-2.3)^2)};
       \addplot [gray!140,very thick,smooth,domain=2.2:2.3,samples=100]{-sqrt(6.017-6.017/.01*(x-2.3)^2)};
       \addplot [gray!140,very thick,smooth,domain=2.3:2.4,samples=100,dashed]{sqrt(6.017-6.017/.01*(x-2.3)^2)};
       \addplot [gray!140,very thick,smooth,domain=2.3:2.4,samples=100,dashed]{-sqrt(6.017-6.017/.01*(x-2.3)^2)};

	\addplot [gray!140,very thick,smooth,domain=2.6:2.7,samples=100]{sqrt(11.136-11.136/.01*(x-2.7)^2)};
       \addplot [gray!140,very thick,smooth,domain=2.6:2.7,samples=100]{-sqrt(11.136-11.136/.01*(x-2.7)^2)};
       \addplot [gray!140,very thick,smooth,domain=2.7:2.8,samples=100,dashed]{sqrt(11.136-11.136/.01*(x-2.7)^2)};
       \addplot [gray!140,very thick,smooth,domain=2.7:2.8,samples=100,dashed]{-sqrt(11.136-11.136/.01*(x-2.7)^2)};


    %fill in                        
            	\addplot [name path=A,domain=2:3,draw=none] {16-19*x+8*x^2-x^3};   
            	\addplot [name path=B,domain=2:3,draw=none] {-16+19*x-8*x^2+x^3};
		\addplot [name path=C,domain=1.9:2.1,draw=none,samples=100] {sqrt(4-4/.01*(x-2)^2)};   
            	\addplot [name path=D,domain=1.9:2.1,draw=none,samples=100] {-sqrt(4-4/.01*(x-2)^2)};
		\addplot [name path=E,domain=2.9:3.1,draw=none,samples=100] {sqrt(16-16/.01*(x-3)^2)};   
            	\addplot [name path=F,domain=2.9:3.1,draw=none,samples=100] {-sqrt(16-16/.01*(x-3)^2)};
            	\addplot [fillp] fill between[of=A and B];
	        \addplot [fillp] fill between[of=C and D];
	        \addplot [fillp!60] fill between[of=E and F];
	        
	        %fill in middle                       
            	\addplot [name path=G,domain=2.3:2.7,draw=none] {16-19*x+8*x^2-x^3};   
            	\addplot [name path=H,domain=2.3:2.7,draw=none] {-16+19*x-8*x^2+x^3};
		\addplot [name path=I,domain=2.2:2.4,draw=none,samples=100] {sqrt(6.017-6.017/.01*(x-2.3)^2)};   
            	\addplot [name path=J,domain=2.2:2.4,draw=none,samples=100] {-sqrt(6.017-6.017/.01*(x-2.3)^2)};
		\addplot [name path=K,domain=2.6:2.8,draw=none,samples=100] {sqrt(11.136-11.136/.01*(x-2.7)^2)};   
            	\addplot [name path=L,domain=2.6:2.8,draw=none,samples=100] {-sqrt(11.136-11.136/.01*(x-2.7)^2)};
            	\addplot [gray!60] fill between[of=G and H];
	        \addplot [gray!60] fill between[of=I and J];
	        \addplot [gray!50] fill between[of=K and L];


    
    \node[penColor] at (axis cs:2.5,4) {$y=f(x)$};
  \end{axis}
\end{tikzpicture}
\end{image}
%%%%%%%%%%%%%%

\paragraph{Step 2: Approximate}
We have seen how to find the surface area of a frustrum, so we should thus approximate each slice as a frustrum.

%%%%%%%%%%
\begin{image}
\begin{tikzpicture}
  \begin{axis}[
      xmin=1.8, xmax=3.2,
      domain=1.5:3.5,
      ymin=-4.5, ymax=4.5,
      clip=false,
      axis lines =center,
      xlabel=$x$, ylabel=$y$, every axis y label/.style={at=(current axis.above origin),anchor=south},
      every axis x label/.style={at=(current axis.right of origin),anchor=west},
      axis on top,
      xtick={2,3},
      xticklabels={$a$,$b$}
    ]
  

    \addplot [penColor,very thick,smooth]{16-19*x+8*x^2-x^3};
    \addplot [penColor2,very thick,smooth,domain=2:3]{16-19*x+8*x^2-x^3};
    \addplot [penColor2,very thick,smooth,domain=2:3]{-16+19*x-8*x^2+x^3};
    
    % top ellipses
     \addplot [penColor2,very thick,smooth,domain=2.9:3.1,samples=100]{sqrt(16-16/.01*(x-3)^2)};
       \addplot [penColor2,very thick,smooth,domain=2.9:3.1,samples=100]{-sqrt(16-16/.01*(x-3)^2)};
       %connect edges of top <-annoying numerical approximation
          \addplot[penColor2, very thick] plot coordinates {(3.1,-.2) (3.1,.2)};  
            
       %bottom ellipses 
       \addplot [penColor2,very thick,smooth,domain=1.9:2,samples=100]{sqrt(4-4/.01*(x-2)^2)};
       \addplot [penColor2,very thick,smooth,domain=1.9:2,samples=100]{-sqrt(4-4/.01*(x-2)^2)};
             
       %intermediate ellipses
       \addplot [gray!140,very thick,smooth,domain=2.2:2.3,samples=100]{sqrt(6.017-6.017/.01*(x-2.3)^2)};
       \addplot [gray!140,very thick,smooth,domain=2.2:2.3,samples=100]{-sqrt(6.017-6.017/.01*(x-2.3)^2)};
       
	\addplot [gray!140,very thick,smooth,domain=2.6:2.7,samples=100]{sqrt(11.136-11.136/.01*(x-2.7)^2)};
       \addplot [gray!140,very thick,smooth,domain=2.6:2.7,samples=100]{-sqrt(11.136-11.136/.01*(x-2.7)^2)};
       \addplot [gray!140,very thick,smooth,domain=2.7:2.8,samples=100,dashed]{sqrt(11.136-11.136/.01*(x-2.7)^2)};
       \addplot [gray!140,very thick,smooth,domain=2.7:2.8,samples=100,dashed]{-sqrt(11.136-11.136/.01*(x-2.7)^2)};


    %fill in                        
            	\addplot [name path=A,domain=2:3,draw=none] {16-19*x+8*x^2-x^3};   
            	\addplot [name path=B,domain=2:3,draw=none] {-16+19*x-8*x^2+x^3};
		\addplot [name path=C,domain=1.9:2.1,draw=none,samples=100] {sqrt(4-4/.01*(x-2)^2)};   
            	\addplot [name path=D,domain=1.9:2.1,draw=none,samples=100] {-sqrt(4-4/.01*(x-2)^2)};
		\addplot [name path=E,domain=2.9:3.1,draw=none,samples=100] {sqrt(16-16/.01*(x-3)^2)};   
            	\addplot [name path=F,domain=2.9:3.1,draw=none,samples=100] {-sqrt(16-16/.01*(x-3)^2)};
            	\addplot [fillp] fill between[of=A and B];
	        \addplot [fillp] fill between[of=C and D];
	        \addplot [fillp!60] fill between[of=E and F];
	        
	        %fill in middle                       
            	\addplot [name path=G,domain=2.3:2.7,draw=none] {16-19*x+8*x^2-x^3};   
            	\addplot [name path=H,domain=2.3:2.7,draw=none] {-16+19*x-8*x^2+x^3};
		\addplot [name path=I,domain=2.2:2.4,draw=none,samples=100] {sqrt(6.017-6.017/.01*(x-2.3)^2)};   
            	\addplot [name path=J,domain=2.2:2.4,draw=none,samples=100] {-sqrt(6.017-6.017/.01*(x-2.3)^2)};
		\addplot [name path=K,domain=2.6:2.8,draw=none,samples=100] {sqrt(11.136-11.136/.01*(x-2.7)^2)};   
            	\addplot [name path=L,domain=2.6:2.8,draw=none,samples=100] {-sqrt(11.136-11.136/.01*(x-2.7)^2)};
            	\addplot [gray!60] fill between[of=G and H];
	        \addplot [gray!60] fill between[of=I and J];
	        \addplot [gray!50] fill between[of=K and L];

%labels
    \addplot[blue,dashed,->] plot coordinates {(2.3,0) (2.3,2.453)};
    \addplot[blue,dashed,->] plot coordinates {(2.7,0) (2.7,3.337)};
    \addplot[red,ultra thick] plot coordinates {(2.3,2.453) (2.7,3.337)};

    \node[blue,anchor=north] at (axis cs:2.4,1.4) {$r_1$};
    \node[blue,anchor=west] at (axis cs:2.7,.6) {$r_2$};
    \node[red,anchor=south] at (axis cs:2.5,3.1) {$\Delta s$};


    
    \node[penColor] at (axis cs:2,2.8) {$y=f(x)$};
  \end{axis}
\end{tikzpicture}
\end{image}%%%%%%%%%%%%%%



%%%%%ORIGINAL PICTURE%%%%%
%\begin{image}
%\begin{tikzpicture}
%  \begin{axis}[
%      xmin=1.5, xmax=3.5,
%      domain=1.5:3.5,
%      ymin=-4.5, ymax=4.5,
%      clip=false,
%      axis lines =center,
%      xlabel=$x$, ylabel=$y$, every axis y label/.style={at=(current axis.above origin),anchor=south},
%      every axis x label/.style={at=(current axis.right of origin),anchor=west},
%      axis on top,
%      xtick={2,2.5},
%      xticklabels={$x$,$x+\d x$}
%    ]
%    \addplot [penColor,very thick,smooth,domain=2:2.5,fill=fill1!50!white,draw=none] {16-19*x+8*x^2-x^3} \closedcycle;
%    \addplot [penColor,very thick,smooth,domain=2:2.5,fill=fill1!50!white,draw=none] {-16+19*x-8*x^2+x^3} \closedcycle;
%
%    \draw[penColor,very thick,fill=fill1!20!white] (axis cs:2,0) ellipse (20 and 200);
%    %\draw[penColor,very thick,fill=fill1] (axis cs:2.5,0) ellipse (20 and 287.5);
%    \draw[penColor,very thick,dashed] (axis cs: 2.5,-2.875) arc (270:90:20 and 287.5);
%    \draw[penColor,very thick,fill=fill1!50!white] (axis cs: 2.5,-2.875) arc (270:450:20 and 287.5);
%
%    \addplot [penColor,very thick,smooth]{16-19*x+8*x^2-x^3};
%    \addplot [penColor,very thick,smooth,domain=2:2.5]{-16+19*x-8*x^2+x^3};
%
%    \addplot[dashed,->] plot coordinates {(2,2) (2.5,2)};
%    \addplot[dashed,->] plot coordinates {(2.5,2) (2.5,2.875)};
%    \addplot[->,ultra thick, penColor2] plot coordinates {(2,2) (2.5,2.875)};
%
%    %\node[anchor=north] at (axis cs:2.25,2) {$\d x$};
%    %\node[anchor=west] at (axis cs:2.5,2.4375) {$\d y$};
%    \node[anchor=south] at (axis cs:2.25,2.6875) {$\d s$};
%
%    \addplot[dashed] plot coordinates {(2,0) (2,2)};
%    \addplot[dashed] plot coordinates {(2.5,0) (2.5,2.875)};
%    
%    \node[penColor] at (axis cs:3,3.5) {$f$};
%  \end{axis}
%\end{tikzpicture}
%\end{image}
%%%%%%%%%%%%%%









Thus the surface area, $\Delta SA$ of this frustum is:
\[
\Delta SA = 2\pi\frac{r_1+r_2}{2} \Delta s
\]
Note that there is a value $r$ between $r_1$ and $r_2$ such that $\frac{r_1+r_2}{2} = \frac{2r}{2} = r$, so we write: 

\[
\Delta SA = 2\pi\frac{r_1+r_2}{2} \Delta s
\]

and can find the total approximate surface area by using $n$ frustra by adding together all of the surface areas:

\[
SA = \sum_{k=1}^n \Delta SA = \sum_{k=1}^n 2\pi r \Delta s
\]

\paragraph{Step 3: Integrate}
The formula above has good conceptual meaning, it does not readily pass to an integral quite yet!
and have seen that we can express $\Delta s$ free in terms of either $\Delta x$ or $\Delta y$, which allows us to express the infinitesimal $\d s$ by:

\[
\d s =  \sqrt{1+\left(\frac{\d y}{\d x} \right)^2}\d x \qquad \textrm{ or } \d s =  \sqrt{1+\left(\frac{\d x}{\d y} \right)^2}\d y
\]

Note also that as the slice widths shrink, the value $r$ above approaches the distance that the corresponding slice is away from the axis of rotation.

To make sure that we emphasize this freedom in expressing $\d s$ as well as the inherent geometric results we used to build the surface area, we write:

\begin{formula}
If a piecewise continuously differentiable function from a point $A$ to a point $B$ in the $xy$-plane is revolved about a non-intersecting vertical or horizontal axis, then the surface area of revolution is found using:

\[
SA=\int_{A}^{B} 2 \pi r \d s 
\]

where the radius $r$ is the distance from the axis of rotation to the slice and $\d s$ is the slant height of the slice.

\end{formula}

\begin{remark} What does this mean?

To compute this surface area, we first choose to express either:

\[
\d s =  \sqrt{1+\left(\frac{\d y}{\d x} \right)^2}\d x \quad \textrm{ or } \quad \d s =  \sqrt{1+\left(\frac{\d x}{\d y} \right)^2}\d y
\]

We then have to express the distance $r$ in terms of the variable of integration. This will always be a vertical or horizontal distance, which can be computed just as we have been doing in previous sections!  
\end{remark}

Using the remark, and letting $A=(a,c)$ and $B=(b,d)$ we can therefore write:

\begin{image}
\begin{tikzpicture}
  \begin{axis}[
      xmin=1.5, xmax=3.5,
      domain=1.5:3.5,
      ymin=-4.5, ymax=4.8,
      clip=false,
      axis lines =center,
      xlabel=$x$, ylabel=$y$, every axis y label/.style={at=(current axis.above origin),anchor=south},
      every axis x label/.style={at=(current axis.right of origin),anchor=west},
      axis on top,
      xtick={2,3},
      xticklabels={$a$,$b$},
      ytick={2,4},
      yticklabels={$c$,$d$}
    ]
  

    \addplot [penColor,very thick,smooth]{16-19*x+8*x^2-x^3};
    \addplot [penColor2,very thick,smooth,domain=2:3]{16-19*x+8*x^2-x^3};
    \addplot [penColor2,very thick,smooth,domain=2:3]{-16+19*x-8*x^2+x^3};
    
    % top ellipses
     \addplot [penColor2,very thick,smooth,domain=2.9:3.1,samples=100]{sqrt(16-16/.01*(x-3)^2)};
       \addplot [penColor2,very thick,smooth,domain=2.9:3.1,samples=100]{-sqrt(16-16/.01*(x-3)^2)};
       %connect edges of top <-annoying numerical approximation
          \addplot[penColor2, very thick] plot coordinates {(3.1,-.2) (3.1,.2)};  
            
       %bottom ellipses 
       \addplot [penColor2,very thick,smooth,domain=1.9:2,samples=100]{sqrt(4-4/.01*(x-2)^2)};
       \addplot [penColor2,very thick,smooth,domain=1.9:2,samples=100]{-sqrt(4-4/.01*(x-2)^2)};
       \addplot [penColor2,very thick,smooth,domain=2:2.1,samples=100,dashed]{sqrt(4-4/.01*(x-2)^2)};
       \addplot [penColor2,very thick,smooth,domain=2:2.1,samples=100,dashed]{-sqrt(4-4/.01*(x-2)^2)};
       
       
      \addplot [penColor2,very thick,smooth,domain=2:3]{-16+19*x-8*x^2+x^3};

    %fill in                        
            	\addplot [name path=A,domain=2:3,draw=none] {16-19*x+8*x^2-x^3};   
            	\addplot [name path=B,domain=2:3,draw=none] {-16+19*x-8*x^2+x^3};
		\addplot [name path=C,domain=1.9:2.1,draw=none,samples=100] {sqrt(4-4/.01*(x-2)^2)};   
            	\addplot [name path=D,domain=1.9:2.1,draw=none,samples=100] {-sqrt(4-4/.01*(x-2)^2)};
		\addplot [name path=E,domain=2.9:3.1,draw=none,samples=100] {sqrt(16-16/.01*(x-3)^2)};   
            	\addplot [name path=F,domain=2.9:3.1,draw=none,samples=100] {-sqrt(16-16/.01*(x-3)^2)};
            	\addplot [fillp] fill between[of=A and B];
	        \addplot [fillp] fill between[of=C and D];
	        \addplot [fillp!60] fill between[of=E and F];

\addplot[color=penColor5,fill=penColor2,only marks,mark=*] coordinates{(2,2)};
\addplot[color=penColor5,fill=penColor2,only marks,mark=*] coordinates{(3,4)};

    \node[penColor2] at (axis cs:2,2.5) {$(a,c)$};
    \node[penColor2] at (axis cs:3,4.5) {$(b,d)$};
  \end{axis}
\end{tikzpicture}
\end{image}



\begin{formula}
If a piecewise continuously differentiable function from a point $(a,c)$ to a point $(c,d)$ in the $xy$-plane is revolved about a non-intersecting vertical or horizontal axis, then the surface area of revolution is found using:

\[
SA=\int_{x=\answer[given]{a}}^{x=\answer[given]{b}} 2 \pi r \answer[given]{\sqrt{1+\left(\frac{\d y}{\d x} \right)^2}} \d x \quad \textrm{ or }  \quad SA=\int_{y=\answer[given]{c}}^{y=\answer[given]{d}} 2 \pi r \answer[given]{\sqrt{1+\left(\frac{\d x}{\d y} \right)^2}} \d y
\]

where the radius $r$ is the distance from the axis of rotation to the slice at $(x,y)$.

\end{formula}


An important concept to note is that the slice is located at a point $(x,y)$ on the curve.  The choice of variable of integration may require that we express either $x$ or $y$ in terms of the other by using the equation that describes the curve.  We will see this in the following examples.


\begin{example}
Consider the surface of revolution found by revolving $f(x) = \sqrt{x}$ from $x=1$ to $x=4$ around the $x$-axis.  We set up an integral with respect to $x$ and an integral with respect to $y$ that gives its surface area.
  
  \begin{explanation}
    
    We begin by considering looking at a picture, 
    
      \begin{image}
      \begin{tikzpicture}
        \begin{axis}[
            xmin=-.3, xmax=4.3,
            domain=0:2.4,
            ymin=-.4, ymax=2.3,
            clip=false,
            axis lines =center,
            xlabel=$x$, ylabel=$y$, every axis y label/.style={at=(current axis.above origin),anchor=south},
            every axis x label/.style={at=(current axis.right of origin),anchor=west},
            axis on top,
          ]
      
                          
          \addplot [penColor,very thick,smooth,samples=200,domain=1:4]{sqrt(x)};
         
          \addplot[ultra thick, penColor2] plot coordinates {(1.4,1.18) (2.3,1.52)};
          
          %\node[anchor=north] at (axis cs:.5,.63) {$\d x$};
          %\node[anchor=west] at (axis cs:2.5,2.4375) {$\d y$};
          \node[penColor2,anchor=south] at (axis cs:1.8,1.5) {$\Delta s$};
                    
          \node[penColor] at (axis cs:3,2) {$y=\sqrt{x}$};
            	 \addplot[color=penColor,fill=penColor,only marks,mark=*] coordinates{(1,1)};
		\addplot[color=penColor,fill=penColor,only marks,mark=*] coordinates{(4,2)};

     		\addplot[color=penColor2,fill=penColor2,only marks,mark=*] coordinates{(1.4,1.18)};
		\addplot[color=penColor2,fill=penColor2,only marks,mark=*] coordinates{(2.3,1.52)};
		
		 \addplot[thick, penColor2] plot coordinates {(2,0) (2,1.41)};
		    \node[penColor2] at (axis cs:2.2,.7) {$r$};
        \end{axis}
      \end{tikzpicture}
    \end{image}

    As a brief aside, note that this slice gives rise to the following frustrum when revolved about the $x$-axis:
    
    \begin{image}
      \begin{tikzpicture}
        \begin{axis}[
            xmin=-.5, xmax=1.5,
            domain=0:1,
            ymin=-1.5, ymax=1.5,
            clip=false,
            axis lines =center,
             xtick={2,3},
      xticklabels={$a$,$b$},
      ytick={2,4},
      yticklabels={$c$,$d$},
                  axis on top,
          ]
          \addplot [fill=fill1!50!white,draw=none,domain=.4:.6] {sqrt(x)} \closedcycle;
          \addplot [fill=fill1!50!white,draw=none,domain=.4:.6] {-sqrt(x)} \closedcycle;
          
          \draw[penColor,very thick,fill=fill1!20!white] (axis cs:.4,0) ellipse (7 and 63);
          \draw[penColor,very thick,dashed] (axis cs: .6,-.77) arc (270:90:7 and 77);
          \draw[penColor,very thick,fill=fill1!50!white] (axis cs: .6,-.77) arc (270:450:7 and 77);
                    
          \addplot [penColor,very thick,smooth,samples=100]{sqrt(x)};
          \addplot [penColor,very thick,smooth,domain=.4:.6]{-sqrt(x)};
          
         
          \addplot[ultra thick, penColor2] plot coordinates {(.4,.63) (.6,.77)};
          
          %\node[anchor=north] at (axis cs:.5,.63) {$\d x$};
          %\node[anchor=west] at (axis cs:2.5,2.4375) {$\d y$};
          \node[penColor2, anchor=south] at (axis cs:.5,.8) {$\Delta s$};
          

        
        \end{axis}
      \end{tikzpicture}
    \end{image}
    
    Let's first set up the integral with respect to $x$.  In order to do this, we choose:
    
    \begin{multipleChoice}
    \choice[correct]{$\d s =  \sqrt{1+\left(\frac{\d y}{\d x} \right)^2}\d x$}
    \choice{$\d s =  \sqrt{1+\left(\frac{\d x}{\d y} \right)^2}\d y$}
    \end{multipleChoice}
    
    Note that $r$ here is a vertical distance that must be expressed in terms of $x$.  Since the slice is located at a point $(x,y)$ on the curve $y=\sqrt{x}$:
    
    \begin{multipleChoice}
    \choice{$y_{top}=y$}
    \choice[correct]{$y_{top}=\sqrt{x}$}
    \choice{$y_{top}=x$}
    \choice{$y_{top}=0$}
    \end{multipleChoice}
    
    \begin{multipleChoice}
    \choice{$y_{bot}=y$}
    \choice{$y_{bot}=\sqrt{x}$}
    \choice{$y_{bot}=x$}
     \choice[correct]{$y_{bot}=0$}
    \end{multipleChoice}

Thus, $r= \answer[given]{\sqrt{x}}$.
     
Now, write with me:
    \[
    \frac{\d y}{\d x} = \answer[given]{\frac{x^{-1/2}}{2}}
    \]
    So
    \begin{align*}
      SA &= \int_{x=\answer{1}}^{x=\answer{4}} 2\pi \sqrt{x} \sqrt{1 + \left(\answer[given]{\frac{x^{-1/2}}{2}}\right)^2} \d x\\
      &= \int_1^4 2\pi \sqrt{x} \sqrt{1 + \frac{1}{4x}} \d x\\
      &= \int_1^4 2\pi \sqrt{x+\frac{1}{4}} \d x\\
      &= \eval{\answer[given]{\frac{4\pi}{3}\left(x+\frac{1}{4}\right)^{3/2}}}_1^4\\
      &= \frac{4\pi}{3}\left(\frac{17}{4}\right)^{3/2}  - \frac{4\pi}{3}\left(\frac{5}{4}\right)^{3/2} \\
      &= \frac{\pi}{6} \left((17)^{3/2}-(5)^{3/2} \right) \textrm{after some algebra} 
    \end{align*}
  \end{explanation}
\end{example}

\begin{example} Now, let's set up the surface are of the region in the previous example as an integral with respect to $y$. 

\begin{explanation}
 In order to do this, we choose:
    
    \begin{multipleChoice}
    \choice{$\d s =  \sqrt{1+\left(\frac{\d y}{\d x} \right)^2}\d x$}
    \choice[correct]{$\d s =  \sqrt{1+\left(\frac{\d x}{\d y} \right)^2}\d y$}
    \end{multipleChoice}
    
    Note that $r$ here is a vertical distance that must be expressed in terms of $y$.  Since the slice is located at a point $(x,y)$ on the curve $y=\sqrt{x}$:
    
    \begin{multipleChoice}
    \choice[correct]{$y_{top}=y$}
    \choice{$y_{top}=y^2$}
    \choice{$y_{top}=0$}
    \end{multipleChoice}
    
\begin{feedback}
Note that the slice is at a given $y$; we do not write $x=y^2$ and use this because the radius $r$ is just the distance from the $x$-axis to the $y$-value on the curve.  The expression $y^2$ actually gives the $x$-coordinate of the slice! 
\end{feedback}

    \begin{multipleChoice}
    \choice{$y_{bot}=y$}
    \choice{$y_{bot}=y^2$}
    \choice[correct]{$y_{bot}=0$}
    \end{multipleChoice}

Thus, $r= \answer[given]{y}$.
     
Now, since we have $x=y^2$, write with me:
    \[
    \frac{\d x}{\d y} = \answer[given]{2y}
    \]
    So
    \begin{align*}
      SA &= \int_{y=\answer{1}}^{y=\answer{2}} 2\pi y \sqrt{1 + \left(\answer[given]{2y}\right)^2} \d x\\
      &= \int_1^2 2\pi y \sqrt{1 +4y^2} \d x\\
      \end{align*}
      
      This integral can be computed using the substitution $u=1+4y^2$.  Working out the details (which you should do on your own), gives:
      
      \begin{align*}
     SA &= \int_{u=\answer[given]{5}}^{u=\answer[given]{17}} \frac{\pi}{4} u^{1/2} \d u \\ 
     &= \eval{\answer[given]{\frac{\pi}{6} u^{3/2}}}_5^{17}\\
     &= \frac{\pi}{6} \left((17)^{3/2}-(5)^{3/2}  \right)
    \end{align*}
  \end{explanation}
  \end{example}
  
  \begin{question}
  The final answers in the previous two examples are:
  \begin{multipleChoice}
    \choice[correct]{equal}
    \choice{not equal}
  \end{multipleChoice}
  \begin{feedback}
    Since they are computing the same surface area, they are necessarily equal. While we should expect this, it's worth noting that the surface area of the solid of revolution is a geometric quantity whose value exists independently of the coordinates used to solve the problem.  

  \end{feedback}
\end{question}




%%%%%%%
%
%On the other hand, if $f^{-1}$ is differentiable on
%an open interval containing $[a,b]$ where $\dd[x]{y}$ is also continuous on
%$[a,b]$, then the area of the surface of revolution formed by
%revolving the graph of $x=f^{-1}(y)$, where $f^{-2}(y)\ge 0$, about the $y$-axis
%is
%\[
%\text{Area} = \int_a^b 2\pi f^{-1}(y)\sqrt{1+\left(\dd[x]{y}\right)^2}\d x.
%\]
%Let's see an example:
%
%\begin{example}
%  Consider the surface of revolution found by revolving $f(x) = x^2$
%  around the $y$-axis. Find its area on the interval when $0\le y\le
%  1$.
%  \begin{explanation}
%    Let's look at a picture, with our infinitesimal frustum added:
%    \begin{image}
%      \begin{tikzpicture}
%        \begin{axis}[
%            xmin=-1.5, xmax=1.5,
%            domain=-1:1,
%            ymin=-.5, ymax=1.5,
%            clip=false,
%            axis lines =center,
%            xlabel=$x$, ylabel=$y$, every axis y label/.style={at=(current axis.above origin),anchor=south},
%            every axis x label/.style={at=(current axis.right of origin),anchor=west},
%            axis on top,
%          ]
%          \addplot [fill=fill1!50!white,draw=none,domain=-.77:.77] {.6} \closedcycle;
%          \addplot [fill=white,draw=none,domain=-.77:.77] {.4} \closedcycle;
%          \addplot [fill=white,draw=none,domain=-.77:.77] {x^2} \closedcycle;
%          
%          \draw[penColor,very thick,fill=fill1!20!white] (axis cs: 0,.6) ellipse (77 and 7);
%          \draw[penColor,very thick,fill=fill1!50!white] (axis cs:.63,.4) arc (360:180:63 and 7);
%          \draw[penColor,very thick,dashed,fill=fill1!50!white] (axis cs: .63,.4) arc (0:180:63 and 7);
%          \addplot [penColor,very thick,smooth]{x^2};
%                    
%          \addplot[dashed,->] plot coordinates {(.63,.4) (.63,.6)};
%          \addplot[dashed,->] plot coordinates {(.63,.6) (.77,.6)};
%          \addplot[->,ultra thick, penColor2] plot coordinates {(.63,.4) (.77,.6)};
%          
%          %\node[anchor=south] at (axis cs:.7 ,.63) {$\d x$};
%          %\node[anchor=west] at (axis cs:2.5,2.4375) {$\d y$};
%          \node[anchor=west] at (axis cs:.7,.5) {$\d s$};
%          
%          \addplot[dashed] plot coordinates {(0,.6) (.77,.6)};
%          \addplot[dashed] plot coordinates {(0,.4) (.63,.4)};
%          
%          \node[penColor] at (axis cs:.8,1) {$f$};
%        \end{axis}
%      \end{tikzpicture}
%    \end{image}
%    When $f$ is restricted to nonnegative numbers, 
%    \[
%    f^{-1}(y) = \answer[given]{\sqrt{y}}
%    \]
%    and
%    \[
%    \dd[x]{y} = \answer[given]{\frac{y^{-1/2}}{2}}
%    \]
%    So
%    \[
%    \text{Area} = \int_0^1 2\pi \sqrt{y} \sqrt{1 + \left(\answer[given]{\frac{y^{-1/2}}{2}}\right)^2} \d y
%    \]
%    but this is the integral we already computed, just with all the $y$'s replaced by $x$'s!   
%  \end{explanation}
%\end{example}
%
%
%


As our final example, we will compute the surface area of the sphere.

% Adapted from Guichard/Mike Wills material

%THIS EXAMPLE IS NO LONGER TYPESETTING...NEED TO FIX THIS!
%\begin{example}
%  Compute the surface area of a sphere of radius $R$. 
%\begin{explanation}
%  The sphere can be obtained by rotating the graph of
%  $f(x)=\sqrt{R^2 - x^2}$ about the $x$-axis.
%  
%    \begin{image}
%      \begin{tikzpicture}
%        \begin{axis}[
%            xmin=-1.2, xmax=1.2,
%            domain=-1:1,
%            ymin=-1.2, ymax=1.2,
%            clip=false,
%            xtick = {-1,1},
%            xticklabels = {$-R$,$R$},
%            ytick = {-1,1},
%            yticklabels = {$-R$,$R$},
%            axis lines =center,
%            xlabel=$x$, ylabel=$y$, every axis y label/.style={at=(current axis.above origin),anchor=south},
%            every axis x label/.style={at=(current axis.right of origin),anchor=west},
%            axis on top,
%          ]
%          \addplot [fill=fill1!50!white,draw=none,domain=.4:.6] {sqrt(1-x^2)} \closedcycle;
%          \addplot [fill=fill1!50!white,draw=none,domain=.4:.6] {-sqrt(1-x^2)} \closedcycle;
%          
%          \draw[penColor,very thick,fill=fill1!20!white] (axis cs:.4,0) ellipse (7 and 92);
%          \draw[penColor,very thick,dashed] (axis cs: .6,-.8) arc (270:90:7 and 80);
%          \draw[penColor,very thick,fill=fill1!50!white] (axis cs: .6,-.8) arc (270:450:7 and 80);
%                    
%          %\addplot [penColor,very thick,smooth,samples=100]{sqrt(1-x^2)};
%          \draw[penColor,very thick] (axis cs: 1,0) arc (0:180:100 and 100);
%          \addplot [penColor,very thick,smooth,domain=.4:.6]{-sqrt(1-x^2)};
%          
%          \addplot[dashed,->] plot coordinates {(.4,.92) (.6,.92)};
%          \addplot[dashed,->] plot coordinates {(.6,.92) (.6,.8)};
%          \addplot[->,ultra thick, penColor2] plot coordinates {(.4,.92) (.6,.8)};
%          
%          %\node[anchor=south] at (axis cs:.5,.92) {$\d x$};
%          %\node[anchor=west] at (axis cs:2.5,2.4375) {$\d y$};
%          \node[anchor=north] at (axis cs:.5,.8) {$\d s$};
%          
%          \addplot[dashed] plot coordinates {(.6,0) (.6,.8)};
%          \addplot[dashed] plot coordinates {(.4,0) (.4,.92)};
%          
%          \node[penColor] at (axis cs:-.8,.8) {$f$};
%        \end{axis}
%      \end{tikzpicture}
%    \end{image}
%    
%    
%    Write with me
%    \[
%    f'(x) = \answer[given]{-x/\sqrt{R^2-x^2}},
%    \]
%  so the area of the surface of revolution is given by:
%  \begin{align*}
%    A&=\int_{-R }^R 2\pi \sqrt{R^2 - x^2}\sqrt{1+\answer[given]{\frac{x^2}{R^2-x^2}}}\d x \\
%    &=\int_{-R }^R 2\pi \sqrt{R^2 - x^2}\sqrt{\frac{R^2}{R^2-x^2}}\d x \\
%    &=\int_{-R }^R 2\pi R \d x\\
%    &=\eval{\answer[given]{2 \pi R x}}_{-R}^R\\
%    &=\answer[given]{4\pi R^2}
%  \end{align*}
%\end{explanation}
%\end{example}

%%%NEED EXAMPLE OF REVOLVING ABOUT ANOTHER LINE BUT WON'T BE NECESSARY UNTIL THIS IS GLOBALLY DEPLOYED%%%%%%%%

\section{Final thoughts}
The key formulas in this section are:

\[
SA=\int_{x=a}^{x=b} 2 \pi r \d s \qquad \textrm{ or }  \qquad SA=\int_{y=c}^{y=d} 2 \pi r \d s
\]

We are free to choose the variable of integration here since we can express $\d s$ in terms of either $\d x$ or $\d y$ easily.  Once this choice of variable has been determined, we need to express the radius $r$ of the infinitesimal frustrum and the limits for the integral in terms of the variable of integration.  

This radius $r$ is the distance from the axis of rotation to the slice at $(x,y)$, which is either a horizontal or vertical distance.  We just need to make sure that we express it in terms of the variable of integration appropriately.

Many of the integrals that arise in the context of these problems can be difficult.  Careful differentiation and algebra, as well as a good grasp of integration techniques can be vital when finding surface areas.  As usual, this can be challenging and practice is the key here.

\begin{quote}
``Math is not a spectator sport.  It's not a body of knowledge.  It's not symbols on a page. It's something you play with, something you do'' - Keith Devlin
\end{quote}







\end{document}
