\documentclass{ximera}

\input{../preamble.tex}

\outcome{Probability Problems and Generalizing Ideas}

\title[Break-Ground:]{Birthday Probability and Dice Games}

\begin{document}
\begin{abstract}
  Thinking through a well-known probability problem.
\end{abstract}
\maketitle


The Birthday Problem:  
We can address this problem as follows:

a)  If one person is in the room, what is the probability that that person does not share a birthday with any other person in the room?
b)  Suppose an additional person enters the room.  What is the probability that the two people in the room do not share a birthday?
c)  Suppose the previous two people did not share a birthday, and an additional person enters the room.  What is the probability that among the three people in the room, no pair shares a birthday?
Moving to a general step…
d)  Suppose that the n-1 (where n<365) previous people did not share a birthday, and an additional person enters the room.  What is the probability that among the n people in the room, no pair shares a birthday?
e)  Find the first value for n which makes the probability that no pair shares a birthday less than 50%.  You’ll likely want to use a calculator or computer.  What probability results from this value of n?  [Round to two decimal places.]
Then for this value of n, the probability that at least one pair shares a birthday is 100% minus the percent probability from part e, which results in a greater than 50% chance.
e’)  Repeat part e and the conclusion if we want at least a 99% chance that at least one pair shares a birthday.  Give the n value and the corresponding probabilities.


%\begin{dialogue}
%\item[Devyn] Riley, do you remember when we first starting graphing
%  functions? Like with a ``T-chart?''
%\end{dialogue}

\begin{problem}
  How many people must be in a room so that the probability that at least two people have the same birthday is at least 50%?
  \begin{multipleChoice}
    \choice{$364$}
    \choice{$183$}
    \choice{$75$}
    \choice[correct]{$23$}
    \choice{$11$}
    \choice{$3+5i$}
  \end{multipleChoice}
\end{problem}

If you got that right, you've probably seen the problem before. The answer is pretty counter-intuitive. Let's think about some probability ideas that we can use to approach the problem.

First, let's make some reasonable assumptions: We can simplify the problem by ignoring leap years, and assume that people are equally likely to be born on any day of the year.

Let's change the way we look at the problem to make it easier $\ldots$

\begin{problem}
  In a random group of $n$ people, which of the following is equivalent to $P\[$ at least $2$ people have the same birthday$\]$ ?
  
  \begin{multipleChoice}
    \choice{$1- P\[$ all people have the same birthday$\]$}
    \choice[correct]{$1- P\[$ no two people have the same birthday$\]$}
    \choice{$P\[$ no two people have the same birthday$\] - P\[$ all people have the same birthday$\]$}
    \choice{$\frac{365-n}{365}}
  \end{multipleChoice}
\end{problem}

As Gyorgy Polya, was alleged to have said "If you can't solve a problem, there there is an easier problem that you still can't solve; solve that first."  There is some debate about whether he said "Can't solve" or "Can solve". I think "can't" is funnier, so we will go with that. 

In any case, here is an easier (similar) problem.

\begin{problem}
  If there are only 2 people (Alice and Bert) in the room, what is the probability that they do not share a birthday?
  \begin{multipleChoice}
    \choice[correct]{$\frac{364}{365}$}
    \choice{$\frac{1}{365}$}
    \choice{$\frac{1}{2}$}  
  \end{multipleChoice}
\end{problem}

There are 364 possible (equally likely) days for Bert's birthday that are not the same as Alice's day.

Make it one step harder and see if you can see a pattern emerge....

\begin{problem}
  If there are exactly 3 people (Alice, Bert and Cindy) in the room, what is the probability that they do not share a birthday?
  \begin{multipleChoice}
    \choice[correct]{$\frac{364}{365}\frac{363}{365}$}
    \choice{$\frac{\pi}{365}$}
    \choice{$\frac{1}{3}$}
  \end{multipleChoice}
\end{problem}

First, Bert has to have a birthday different than Alice. Then assuming that is true, Cindy has 363 days that are free for her birthday.
 
 
 Do you see the pattern?
 Now try 20 people.
 
 \begin{problem}
  If there are exactly 20 people (Alice, Bert and Cindy) in the room, what is the probability that they do not share a birthday?
  \begin{multipleChoice}
    \choice[correct]{$\frac{364}{365}\frac{363}{365}\dot\ldots \dot \frac{365-19}{365}$}
    \choice{$\frac{\pi}{365}$}
    \choice{$\frac{1}{3}$}
  \end{multipleChoice}
\end{problem}

%\input{../leveledQuestions.tex}


\end{document}
